\documentclass[10pt,twocolumn]{article}

% ===================== Packages =====================
\usepackage[T1]{fontenc}
\usepackage{lmodern}
\usepackage{microtype}
\usepackage{mathtools,amssymb,amsfonts}
\usepackage{bm}           % bold math
\usepackage{siunitx}
\usepackage{geometry}
\usepackage{hyperref}
\usepackage{booktabs}
\usepackage{enumitem}
\usepackage{algorithm}
\usepackage{algpseudocode}

% ===================== Layout =====================
\geometry{margin=0.68in}
\setlength{\columnsep}{0.22in}
\setlist[itemize]{leftmargin=*,itemsep=2pt,topsep=2pt}
\setlist[enumerate]{leftmargin=*,itemsep=2pt,topsep=2pt}
\setlength{\emergencystretch}{2em} % helps avoid overfull boxes

\hypersetup{
  colorlinks=true,
  linkcolor=blue,
  urlcolor=blue,
  citecolor=blue
}

% ===================== Macros =====================
\newcommand{\R}{\mathbb{R}}
\newcommand{\SO}{\mathrm{SO}}
\newcommand{\diag}{\mathrm{diag}}
\newcommand{\E}{\mathbb{E}}
\newcommand{\I}{\mathbf{I}}
\newcommand{\0}{\mathbf{0}}

% Skew matrix: [v]_x such that [v]_x u = v x u
\newcommand{\skew}[1]{\left[#1\right]_{\times}}

% Quaternions / rotations
\newcommand{\qref}{\mathbf{q}_{\mathrm{ref}}}     % WORLD -> BODY'
\newcommand{\RwB}{\mathbf{R}_{wb}}                % WORLD -> BODY' (matrix of qref)
\newcommand{\RbW}{\mathbf{R}_{bw}}                % BODY' -> WORLD
\newcommand{\dth}{\delta\bm{\theta}}
\newcommand{\bg}{\mathbf{b}_g}
\newcommand{\ba}{\mathbf{b}_a}
\newcommand{\w}{\bm{\omega}}
\newcommand{\gvec}{\mathbf{g}}
\newcommand{\aw}{\mathbf{a}_w}
\newcommand{\accm}{\mathbf{a}_m}
\newcommand{\magm}{\mathbf{m}_m}
\newcommand{\eThree}{\mathbf{e}_3}
\newcommand{\Ts}{\Delta t}
\newcommand{\Hs}{H_s}

\newcommand{\norm}[1]{\left\lVert #1 \right\rVert}

% ===================== Title =====================
\title{\vspace{-0.35in}\bfseries
Kalman3D\_Wave\_2: Broadband Oscillator Sea-State + Quaternion Error-State Attitude\\
Full Mathematics, Discretization, and Implementation-Traceable Algorithms
}
\author{Mikhail Grushinskiy}
\date{\vspace{-0.35in}}

\begin{document}
\maketitle
\vspace{-0.25in}

\begin{abstract}
This document describes a marine IMU fusion method that combines:
(i) quaternion error-state attitude estimation (qMEKF-style, right-multiply corrections),
(ii) a broadband sea-motion model implemented as a bank of damped oscillators producing displacement, velocity, and acceleration in WORLD, and
(iii) robust tilt-compensated magnetometer yaw correction using a direction-only residual with $180^\circ$ ambiguity handling.
We provide continuous-time models, discrete-time propagation (including closed-form oscillator transitions for under/critical/over-damped regimes and Simpson-rule noise integration),
accelerometer and magnetometer measurement models and Jacobians, NIS gating, a Joseph-form covariance update, warmup/bias gating logic, and an optional stabilization approximation that removes cross-axis covariances to force per-axis independence.
\end{abstract}

% ==============================================================================
\section{Frames, Conventions, and Update Order}

\paragraph{Frames.}
WORLD is NED by default, with $+Z$ pointing \emph{down}. BODY' is a \emph{virtual} un-heeled body frame. (Heel handling is a separate rotation; the filter may ``de-heel'' IMU vectors and provide a ``boat quaternion'' externally.)

\paragraph{Stored quaternion.}
The filter stores a reference quaternion
\[
\qref:\ \text{WORLD}\rightarrow\text{BODY'}.
\]
Its rotation matrix is $\RwB=\mathbf{R}(\qref)\in\SO(3)$ and $\RbW=\RwB^\top$.

\paragraph{Error-state correction.}
Attitude error is a small rotation vector $\dth\in\R^3$, applied by right multiplication:
\begin{equation}
\qref \leftarrow \qref \otimes \delta q,\qquad
\delta q=\exp\!\left(\tfrac12\,\dth\right),\qquad
\dth\leftarrow \0.
\label{eq:q_correction}
\end{equation}
All Jacobians in this document assume this correction convention.

\paragraph{Per-sample update order.}
For each IMU sample:
\begin{enumerate}
\item Time update: propagate $\qref$, propagate oscillator states, propagate covariance.
\item Accelerometer update: specific-force measurement update (with warmup/bias gating).
\item Magnetometer update: delayed, direction-only yaw correction (often restricted to base states).
\item Stability: symmetrize, clamp diagonal, PSD projection; optionally remove cross-axis covariances.
\end{enumerate}

\paragraph{Implementation mapping (function names).}
For traceability, the key functions are:
\begin{itemize}
\item \texttt{time\_update}
\item \texttt{measurement\_update\_acc\_only}
\item \texttt{measurement\_update\_mag\_only}
\item \texttt{discretize\_osc\_axis\_} (per-axis oscillator $\Phi,Q_d$)
\item \texttt{joseph\_update3\_}
\end{itemize}

% ==============================================================================
\section{State Vector and Dimension}

Let $K$ be the number of oscillator modes (compile-time constant \texttt{KMODES}). Each mode has displacement $\mathbf{p}_k\in\R^3$ and velocity $\mathbf{v}_k\in\R^3$ in WORLD.

\subsection{State definition}
The error-state vector is
\begin{equation}
\mathbf{x}=
\begin{bmatrix}
\dth\\
(\bg)\\
\mathbf{p}_1\\ \mathbf{v}_1\\
\vdots\\
\mathbf{p}_K\\ \mathbf{v}_K\\
(\ba)
\end{bmatrix},
\label{eq:state}
\end{equation}
where gyro bias $\bg$ and accelerometer bias $\ba$ are optional compile-time blocks.

\subsection{Dimension}
The dimension is
\begin{equation}
N_X =
\underbrace{\begin{cases}
3 & \text{attitude error only}\\
6 & \text{attitude error + gyro bias}
\end{cases}}_{\text{base}}
+
\underbrace{6K}_{\text{wave block}}
+
\underbrace{\begin{cases}
3 & \text{+ accel bias}\\
0 & \text{no accel bias}
\end{cases}}_{\text{accel bias}}.
\label{eq:nx}
\end{equation}
With defaults $K=3$, gyro bias on, accel bias on: $N_X=27$.

\subsection{Wave block ordering (critical)}
For each mode $k$, the stored order is \emph{p then v}:
\[
\mathbf{x}_k =
\begin{bmatrix}
p_{k,x}&p_{k,y}&p_{k,z}&v_{k,x}&v_{k,y}&v_{k,z}
\end{bmatrix}^{\!\top}.
\]
This determines how the $6\times 6$ discretization is assembled from per-axis $2\times 2$ blocks.

% ==============================================================================
\section{Continuous-Time Process Models}

\subsection{Attitude and gyro bias (error-state form)}
Let measured angular rate in BODY' be $\w_m$ and estimated gyro bias be $\bg$. Define $\w=\w_m-\bg$.
A standard small-error model consistent with right-multiply correction is
\begin{align}
\dot{\dth} &= -\skew{\w}\,\dth\ -\ \delta\bg\ -\mathbf{n}_g, \label{eq:ct_dth}\\
\dot{\delta\bg} &= \mathbf{n}_{bg}, \label{eq:ct_bg}
\end{align}
with diagonal noise PSDs $\mathbf{Q}_g$ and $\mathbf{Q}_{bg}$ (implementation uses diagonal matrices).

\subsection{Broadband oscillator model (per mode, per axis)}
For each mode $k$ and each axis independently:
\begin{align}
\dot p_k &= v_k, \\
\dot v_k &= -\omega_k^2 p_k - 2\zeta_k\omega_k v_k + \xi_k(t),
\end{align}
with white driving noise $\xi_k$:
\[
\E[\xi_k(t)\xi_k(\tau)] = q_k\,\delta(t-\tau),\qquad q_k\ [\si{m^2/s^5}].
\]
In LTI form for one axis, $\mathbf{s}=[p,v]^\top$:
\[
\dot{\mathbf{s}} = \mathbf{A}\mathbf{s}+\mathbf{G}\xi,\qquad
\mathbf{A}=\begin{bmatrix}0&1\\-\omega^2&-2\zeta\omega\end{bmatrix},\quad
\mathbf{G}=\begin{bmatrix}0\\1\end{bmatrix}.
\]

\subsection{Wave acceleration reconstruction in WORLD}
The oscillator bank produces WORLD wave acceleration
\begin{equation}
\aw=\sum_{k=1}^{K}\left(-\omega_k^2\mathbf{p}_k-2\zeta_k\omega_k\mathbf{v}_k\right).
\label{eq:aw}
\end{equation}

% ==============================================================================
\section{Discrete-Time Propagation}

\subsection{Quaternion propagation}
With $\w=\w_m-\bg$, the quaternion is propagated as
\begin{equation}
\qref_{k+1}=\qref_{k}\otimes\exp\!\left(\tfrac12\,\w\,\Ts\right).
\label{eq:q_prop}
\end{equation}
(Here $\qref_k$ denotes the value of $\qref$ at time step $k$; we do \emph{not} write \texttt{\textbackslash qref\_} in LaTeX to avoid double subscripts.)

\subsection{Base block discretization}
This subsection derives the discrete-time model for the $(\dth,\bg)$ base block.

\paragraph{Rotation integral form.}
Let $\Omega=\skew{\w}$, and define
\[
\mathbf{R}(t)=\exp(-\Omega t),\qquad
\mathbf{B}(t)=-\int_{0}^{t}\mathbf{R}(\tau)\,d\tau.
\]
Then the ``exact-ish'' discrete covariance used in the implementation for the base block is built from
\begin{align}
\mathbf{Q}_{\theta\theta} &= \int_{0}^{\Ts}\mathbf{R}(t)\mathbf{Q}_{g}\mathbf{R}(t)^\top dt
+ \int_{0}^{\Ts}\mathbf{B}(t)\mathbf{Q}_{bg}\mathbf{B}(t)^\top dt, \label{eq:Qtt}\\
\mathbf{Q}_{bb} &= \mathbf{Q}_{bg}\,\Ts, \label{eq:Qbb}\\
\mathbf{Q}_{\theta b} &= \left(\int_{0}^{\Ts}\mathbf{B}(t)\,dt\right)\mathbf{Q}_{bg}. \label{eq:Qtb}
\end{align}
The full base $Q_d$ is
\[
\mathbf{Q}_{AA}=
\begin{bmatrix}
\mathbf{Q}_{\theta\theta} & \mathbf{Q}_{\theta b}\\
\mathbf{Q}_{\theta b}^\top & \mathbf{Q}_{bb}
\end{bmatrix},
\]
followed by symmetrization and PSD projection.

\paragraph{Simpson integration.}
Each integral is approximated by Simpson quadrature:
\[
\int_{0}^{\Ts} f(t)\,dt \approx \frac{\Ts}{6}\bigl(f(0)+4f(\Ts/2)+f(\Ts)\bigr).
\]

\subsection{Oscillator transition $\Phi(t)$ (closed form)}
For one axis (drop $k$ and axis subscripts), let $a=\zeta\omega$.

\paragraph{Critical damping ($\zeta=1$).}
\begin{equation}
\Phi(t)=e^{-a t}
\begin{bmatrix}
1+a t & t\\
-\omega^2 t & 1-a t
\end{bmatrix}.
\label{eq:phi_crit}
\end{equation}

\paragraph{Underdamped ($\zeta<1$).}
Let $\omega_d=\omega\sqrt{1-\zeta^2}$. Then
\begin{equation}
\Phi(t)=e^{-a t}
\begin{bmatrix}
\cos(\omega_d t)+\frac{a}{\omega_d}\sin(\omega_d t) & \frac{1}{\omega_d}\sin(\omega_d t)\\
-\frac{\omega^2}{\omega_d}\sin(\omega_d t) & \cos(\omega_d t)-\frac{a}{\omega_d}\sin(\omega_d t)
\end{bmatrix}.
\label{eq:phi_under}
\end{equation}

\paragraph{Overdamped ($\zeta>1$).}
Let $s=\sqrt{\zeta^2-1}$ and
\[
r_1=-\omega(\zeta-s),\qquad r_2=-\omega(\zeta+s).
\]
One valid closed form is
\begin{equation}
\Phi(t)=\frac{1}{r_2-r_1}
\begin{bmatrix}
r_2 e^{r_1 t} - r_1 e^{r_2 t} & e^{r_2 t}-e^{r_1 t}\\
r_1 r_2 (e^{r_1 t}-e^{r_2 t}) & r_2 e^{r_2 t}-r_1 e^{r_1 t}
\end{bmatrix}.
\label{eq:phi_over}
\end{equation}
(Implementation uses numerically stable forms with \texttt{expm1} for near-equal exponents.)

\subsection{Oscillator discrete covariance $Q_d$ (Simpson)}
With $\dot{\mathbf{s}}=\mathbf{A}\mathbf{s}+\mathbf{G}\xi$ and $\mathbf{G}=[0,1]^\top$,
\begin{equation}
\mathbf{Q}_d=q\int_{0}^{\Ts}\Phi(t)\mathbf{G}\mathbf{G}^\top\Phi(t)^\top dt.
\label{eq:qd_int}
\end{equation}
Because $\mathbf{G}$ selects the second column of $\Phi(t)$, define
\[
\mathbf{u}(t)=
\begin{bmatrix}\Phi_{01}(t)\\\Phi_{11}(t)\end{bmatrix},
\qquad
\mathbf{Q}_d=q\int_{0}^{\Ts}\mathbf{u}(t)\mathbf{u}(t)^\top dt.
\]
Simpson approximation:
\[
\mathbf{Q}_d\approx q\cdot\frac{\Ts}{6}\Bigl(
u(0)u(0)^\top + 4u(\Ts/2)u(\Ts/2)^\top + u(\Ts)u(\Ts)^\top
\Bigr),
\]
followed by symmetrization and PSD projection.

\subsection{$6\times 6$ assembly for one mode (matches p-then-v ordering)}
Given per-axis $2\times 2$ blocks $\Phi_a$ for $a\in\{x,y,z\}$, assemble
\[
\Phi_{6}=
\begin{bmatrix}
A & B\\
C & D
\end{bmatrix},
\quad
A=\diag(\Phi_{00}^x,\Phi_{00}^y,\Phi_{00}^z),\;
B=\diag(\Phi_{01}^x,\Phi_{01}^y,\Phi_{01}^z),
\]
and similarly $C=\diag(\Phi_{10}^a)$, $D=\diag(\Phi_{11}^a)$.
Assemble $Q_{d,6}$ with the same block structure from per-axis $Q_d$ entries.

% ==============================================================================
\section{Accelerometer Measurement Update}

\subsection{Measurement model}
Accelerometer measures specific force in BODY' (after de-heel):
\begin{equation}
\accm = \RwB(\aw-\gvec) + \mathbf{a}_{\mathrm{lever}} + \ba_{\mathrm{term}} + \eta_a,
\label{eq:acc_model}
\end{equation}
where WORLD gravity is $\gvec=[0,0,g]^\top$ (with $+Z$ down).

\paragraph{Lever-arm mean (optional).}
With lever arm $\mathbf{r}$ (BODY'), angular rate $\w$, and angular acceleration $\bm{\alpha}$:
\[
\mathbf{a}_{\mathrm{lever}}\approx \bm{\alpha}\times\mathbf{r} + \w\times(\w\times\mathbf{r}).
\]

\paragraph{Bias term modes (implementation).}
With temperature compensation $\mathbf{k}_a(T-T_0)$:
\begin{itemize}
\item Mode A (bias updates enabled): $\ba_{\mathrm{term}}=\ba+\mathbf{k}_a(T-T_0)$ and $\ba$ is updated.
\item Mode B (bias updates disabled): $\ba_{\mathrm{term}}=\mathbf{k}_a(T-T_0)$, but $\mathrm{Cov}(\ba)$ is injected into innovation as nuisance noise.
\end{itemize}

\subsection{Residual and Jacobians}
Predicted mean $\hat{\accm}$ is the RHS of \eqref{eq:acc_model} without $\eta_a$.
Residual: $\mathbf{r}_a=\accm-\hat{\accm}$.

Let $\mathbf{f}_{\mathrm{cog}}=\RwB(\aw-\gvec)$. Using $\delta(\RwB \mathbf{u})=-\skew{\RwB\mathbf{u}}\dth$:
\begin{equation}
\mathbf{J}_\theta = \frac{\partial \hat{\accm}}{\partial \dth}
= -\skew{\mathbf{f}_{\mathrm{cog}}}.
\label{eq:Jtheta_acc}
\end{equation}
From \eqref{eq:aw}:
\[
\frac{\partial \aw}{\partial \mathbf{p}_k}=-\omega_k^2\I,\qquad
\frac{\partial \aw}{\partial \mathbf{v}_k}=-2\zeta_k\omega_k\I,
\]
so
\[
\frac{\partial \hat{\accm}}{\partial \mathbf{p}_k}=\RwB(-\omega_k^2\I),\qquad
\frac{\partial \hat{\accm}}{\partial \mathbf{v}_k}=\RwB(-2\zeta_k\omega_k\I).
\]
If accel bias updates are enabled, $\partial\hat{\accm}/\partial \ba=\I$.

\subsection{Innovation covariance, NIS gate, Joseph update}
Let $\mathbf{R}_{acc}$ be diagonal accel measurement covariance.
Innovation covariance:
\[
\mathbf{S}=\mathbf{R}_{acc} + \mathbf{J}\mathbf{P}\mathbf{J}^\top.
\]
If the wave block is disabled, missing wave acceleration uncertainty is injected:
\[
\mathbf{S}\leftarrow \mathbf{S} + \RwB\,\Sigma_{\aw}^{(\mathrm{disabled})}\,\RwB^\top.
\]
If accel bias updates are disabled (Mode B), add $\mathbf{P}_{ba,ba}$ to $\mathbf{S}$ as nuisance.

NIS:
\[
\mathrm{NIS}=\mathbf{r}_a^\top \mathbf{S}^{-1}\mathbf{r}_a,
\]
reject if above a gate.

Gain:
\[
\mathbf{K}=\mathbf{P}\mathbf{J}^\top\mathbf{S}^{-1},
\]
then state update $\mathbf{x}\leftarrow \mathbf{x}+\mathbf{K}\mathbf{r}_a$ and Joseph-form covariance update:
\begin{equation}
\mathbf{P}\leftarrow \mathbf{P}
- \mathbf{K}(\mathbf{P}\mathbf{J}^\top)
- (\mathbf{K}(\mathbf{P}\mathbf{J}^\top))^\top
+ \mathbf{K}\mathbf{S}\mathbf{K}^\top.
\label{eq:joseph}
\end{equation}
Finally apply quaternion correction \eqref{eq:q_correction} and clear $\dth$.

% ==============================================================================
\section{Magnetometer Update: Tilt-Compensated Direction-Only}

\subsection{Direction-only measurement}
Let $\mathbf{B}_W$ be the world magnetic reference.
Predicted field in BODY':
\[
\hat{\mathbf{b}}=\RwB\mathbf{B}_W.
\]
Predicted down direction in BODY':
\[
\mathbf{d}=\RwB\eThree,\qquad \eThree=[0,0,1]^\top.
\]
Horizontal projector:
\[
\mathbf{P}_h=\I-\mathbf{d}\mathbf{d}^\top.
\]
Project and normalize:
\[
\mathbf{z}=\frac{\mathbf{P}_h\magm}{\norm{\mathbf{P}_h\magm}},\qquad
\mathbf{h}=\frac{\mathbf{P}_h\hat{\mathbf{b}}}{\norm{\mathbf{P}_h\hat{\mathbf{b}}}}.
\]
Resolve $180^\circ$ ambiguity: if $\mathbf{z}^\top\mathbf{h}<0$, flip $\mathbf{h}\leftarrow-\mathbf{h}$.
Residual: $\mathbf{r}_m=\mathbf{z}-\mathbf{h}$.

\subsection{Finite-difference Jacobian (robust to projection/flip)}
Analytic Jacobians for projection + normalization + flip are easy to break.
A robust implementation-matched choice is symmetric finite differences w.r.t.\ $\dth$ using the \emph{same} correction convention:
\[
\mathbf{J}_\theta[:,i]\approx
\frac{\mathbf{h}(\qref\otimes\exp(\tfrac12\epsilon \mathbf{e}_i))-\mathbf{h}(\qref\otimes\exp(-\tfrac12\epsilon \mathbf{e}_i))}
{2\epsilon},
\]
while enforcing the \emph{same branch} as nominal: if the perturbed $\mathbf{h}$ has negative dot with the nominal $\mathbf{h}$, flip it before differencing.

\subsection{Direction-domain noise and base-only update}
A practical direction noise variance is
\[
\sigma_{\mathrm{dir}}^2 \approx
\frac{\tfrac13\,\mathrm{tr}(\mathbf{R}_{mag})}{\norm{\mathbf{P}_h\magm}^2},
\qquad
\mathbf{S}\approx \sigma_{\mathrm{dir}}^2\I + \mathbf{J}_\theta\mathbf{P}_{\theta\theta}\mathbf{J}_\theta^\top.
\]
Apply NIS gating and then a Joseph update as in \eqref{eq:joseph}.
To avoid coupling magnetometer corrections into wave/bias via cross-covariances, the gain can be restricted to the base states (attitude and optionally gyro bias) by zeroing gain rows for wave and accel-bias indices.

% ==============================================================================
\section{Warmup and Staged Enabling}

Warmup mode is a staged initialization policy:
\begin{itemize}
\item Wave block disabled; wave states and their covariance are reset small.
\item Lever-arm correction can be disabled.
\item Accelerometer bias updates can be disabled (Mode B), but $\mathrm{Cov}(\ba)$ still contributes as nuisance noise.
\item Gyro bias can be learned only under stationarity gating (small motion thresholds).
\item Magnetometer updates are delayed to avoid early transient corruption.
\end{itemize}

% ==============================================================================
\section{Axis-Independent Covariance Approximation}

If enabled, the covariance is forced block-diagonal across axes $x/y/z$ by removing cross-axis blocks.

\subsection{Index sets}
Define axis index sets $\mathcal{I}_x,\mathcal{I}_y,\mathcal{I}_z$ that include all states for a given axis:
\[
\mathcal{I}_a=\{\dth_a,\ (b_{g,a}),\ p_{1,a},v_{1,a},\dots,p_{K,a},v_{K,a},\ (b_{a,a})\},\quad a\in\{x,y,z\}.
\]

\subsection{Enforcement}
After propagation and after each measurement update:
\begin{enumerate}
\item Set $\mathbf{P}_{\mathcal{I}_a,\mathcal{I}_b}\leftarrow\0$ for $a\neq b$.
\item For each axis $a$, symmetrize and PSD-project the submatrix $\mathbf{P}_{\mathcal{I}_a,\mathcal{I}_a}$.
\item Symmetrize full $\mathbf{P}$ and clamp diagonals.
\end{enumerate}
This is a stabilizing approximation (not statistically optimal if true dynamics couple axes).

% ==============================================================================
\section{Algorithms (Implementation-Traceable)}

\begin{algorithm}[t]
\caption{Per-sample update loop (high-level call order)}
\begin{algorithmic}[1]
\Require IMU sample $(\w_m,\accm)$ at period $\Ts$, temperature $T$; optional magnetometer sample $\magm$
\State De-heel IMU vectors if heel compensation is enabled
\State \textbf{TimeUpdate}$(\w_m,\Ts)$ \Comment{\texttt{time\_update}}
\If{warmup mode}
  \State \textbf{WarmupUpdate}$(\accm,\w_m,\Ts)$ \Comment{\texttt{update\_initialization}}
\EndIf
\State \textbf{AccelUpdate}$(\accm,T)$ \Comment{\texttt{measurement\_update\_acc\_only}}
\If{mag enabled and delay satisfied and sample available}
  \State \textbf{MagUpdate}$(\magm)$ \Comment{\texttt{measurement\_update\_mag\_only}}
\EndIf
\State Symmetrize $\mathbf{P}$; clamp diagonal; PSD project as configured
\If{axis-independence enabled}
  \State Zero cross-axis blocks; per-axis PSD project
\EndIf
\end{algorithmic}
\end{algorithm}

\begin{algorithm}[t]
\caption{Oscillator axis discretization (matches \texttt{discretize\_osc\_axis\_})}
\begin{algorithmic}[1]
\Require $\omega,\zeta,q,\Ts$
\State Compute $\Phi(\Ts)$ using under/critical/over-damped closed forms
\State $\mathbf{u}(t)\gets[\Phi_{01}(t),\Phi_{11}(t)]^\top$ at $t\in\{0,\Ts/2,\Ts\}$
\State $\mathbf{Q}_d \gets q\cdot\frac{\Ts}{6}\bigl(u(0)u(0)^\top+4u(\Ts/2)u(\Ts/2)^\top+u(\Ts)u(\Ts)^\top\bigr)$
\State Symmetrize $\mathbf{Q}_d$ and PSD project
\State \Return $(\Phi(\Ts),\mathbf{Q}_d)$
\end{algorithmic}
\end{algorithm}

% ==============================================================================
\section{Checklist (What Must Match the Code)}
\begin{itemize}
\item $\qref$ is WORLD$\rightarrow$BODY' and correction is right-multiply \eqref{eq:q_correction}.
\item WORLD is NED with $+Z$ down; $\gvec=[0,0,g]^\top$.
\item Wave acceleration is computed in WORLD by \eqref{eq:aw}, rotated only inside measurement prediction.
\item Wave state ordering is p(3) then v(3) per mode; $6\times6$ assembly must match.
\item Accel update uses $-\skew{\RwB(\aw-\gvec)}$ for the attitude Jacobian.
\item Mag update is direction-only with horizontal projection and $180^\circ$ ambiguity handling; FD Jacobian keeps a fixed branch.
\item Joseph update uses the three-term form \eqref{eq:joseph} and is followed by quaternion correction.
\item Optional axis-independence zeroes cross-axis blocks in $\mathbf{P}$.
\end{itemize}

\end{document}
